% Options for packages loaded elsewhere
\PassOptionsToPackage{unicode}{hyperref}
\PassOptionsToPackage{hyphens}{url}
%
\documentclass[
]{article}
\usepackage{lmodern}
\usepackage{amssymb,amsmath}
\usepackage{ifxetex,ifluatex}
\ifnum 0\ifxetex 1\fi\ifluatex 1\fi=0 % if pdftex
  \usepackage[T1]{fontenc}
  \usepackage[utf8]{inputenc}
  \usepackage{textcomp} % provide euro and other symbols
\else % if luatex or xetex
  \usepackage{unicode-math}
  \defaultfontfeatures{Scale=MatchLowercase}
  \defaultfontfeatures[\rmfamily]{Ligatures=TeX,Scale=1}
\fi
% Use upquote if available, for straight quotes in verbatim environments
\IfFileExists{upquote.sty}{\usepackage{upquote}}{}
\IfFileExists{microtype.sty}{% use microtype if available
  \usepackage[]{microtype}
  \UseMicrotypeSet[protrusion]{basicmath} % disable protrusion for tt fonts
}{}
\makeatletter
\@ifundefined{KOMAClassName}{% if non-KOMA class
  \IfFileExists{parskip.sty}{%
    \usepackage{parskip}
  }{% else
    \setlength{\parindent}{0pt}
    \setlength{\parskip}{6pt plus 2pt minus 1pt}}
}{% if KOMA class
  \KOMAoptions{parskip=half}}
\makeatother
\usepackage{xcolor}
\IfFileExists{xurl.sty}{\usepackage{xurl}}{} % add URL line breaks if available
\IfFileExists{bookmark.sty}{\usepackage{bookmark}}{\usepackage{hyperref}}
\hypersetup{
  pdftitle={Introduction to R},
  pdfauthor={Qin Wang or},
  hidelinks,
  pdfcreator={LaTeX via pandoc}}
\urlstyle{same} % disable monospaced font for URLs
\usepackage[margin=1in]{geometry}
\usepackage{color}
\usepackage{fancyvrb}
\newcommand{\VerbBar}{|}
\newcommand{\VERB}{\Verb[commandchars=\\\{\}]}
\DefineVerbatimEnvironment{Highlighting}{Verbatim}{commandchars=\\\{\}}
% Add ',fontsize=\small' for more characters per line
\usepackage{framed}
\definecolor{shadecolor}{RGB}{248,248,248}
\newenvironment{Shaded}{\begin{snugshade}}{\end{snugshade}}
\newcommand{\AlertTok}[1]{\textcolor[rgb]{0.94,0.16,0.16}{#1}}
\newcommand{\AnnotationTok}[1]{\textcolor[rgb]{0.56,0.35,0.01}{\textbf{\textit{#1}}}}
\newcommand{\AttributeTok}[1]{\textcolor[rgb]{0.77,0.63,0.00}{#1}}
\newcommand{\BaseNTok}[1]{\textcolor[rgb]{0.00,0.00,0.81}{#1}}
\newcommand{\BuiltInTok}[1]{#1}
\newcommand{\CharTok}[1]{\textcolor[rgb]{0.31,0.60,0.02}{#1}}
\newcommand{\CommentTok}[1]{\textcolor[rgb]{0.56,0.35,0.01}{\textit{#1}}}
\newcommand{\CommentVarTok}[1]{\textcolor[rgb]{0.56,0.35,0.01}{\textbf{\textit{#1}}}}
\newcommand{\ConstantTok}[1]{\textcolor[rgb]{0.00,0.00,0.00}{#1}}
\newcommand{\ControlFlowTok}[1]{\textcolor[rgb]{0.13,0.29,0.53}{\textbf{#1}}}
\newcommand{\DataTypeTok}[1]{\textcolor[rgb]{0.13,0.29,0.53}{#1}}
\newcommand{\DecValTok}[1]{\textcolor[rgb]{0.00,0.00,0.81}{#1}}
\newcommand{\DocumentationTok}[1]{\textcolor[rgb]{0.56,0.35,0.01}{\textbf{\textit{#1}}}}
\newcommand{\ErrorTok}[1]{\textcolor[rgb]{0.64,0.00,0.00}{\textbf{#1}}}
\newcommand{\ExtensionTok}[1]{#1}
\newcommand{\FloatTok}[1]{\textcolor[rgb]{0.00,0.00,0.81}{#1}}
\newcommand{\FunctionTok}[1]{\textcolor[rgb]{0.00,0.00,0.00}{#1}}
\newcommand{\ImportTok}[1]{#1}
\newcommand{\InformationTok}[1]{\textcolor[rgb]{0.56,0.35,0.01}{\textbf{\textit{#1}}}}
\newcommand{\KeywordTok}[1]{\textcolor[rgb]{0.13,0.29,0.53}{\textbf{#1}}}
\newcommand{\NormalTok}[1]{#1}
\newcommand{\OperatorTok}[1]{\textcolor[rgb]{0.81,0.36,0.00}{\textbf{#1}}}
\newcommand{\OtherTok}[1]{\textcolor[rgb]{0.56,0.35,0.01}{#1}}
\newcommand{\PreprocessorTok}[1]{\textcolor[rgb]{0.56,0.35,0.01}{\textit{#1}}}
\newcommand{\RegionMarkerTok}[1]{#1}
\newcommand{\SpecialCharTok}[1]{\textcolor[rgb]{0.00,0.00,0.00}{#1}}
\newcommand{\SpecialStringTok}[1]{\textcolor[rgb]{0.31,0.60,0.02}{#1}}
\newcommand{\StringTok}[1]{\textcolor[rgb]{0.31,0.60,0.02}{#1}}
\newcommand{\VariableTok}[1]{\textcolor[rgb]{0.00,0.00,0.00}{#1}}
\newcommand{\VerbatimStringTok}[1]{\textcolor[rgb]{0.31,0.60,0.02}{#1}}
\newcommand{\WarningTok}[1]{\textcolor[rgb]{0.56,0.35,0.01}{\textbf{\textit{#1}}}}
\usepackage{graphicx,grffile}
\makeatletter
\def\maxwidth{\ifdim\Gin@nat@width>\linewidth\linewidth\else\Gin@nat@width\fi}
\def\maxheight{\ifdim\Gin@nat@height>\textheight\textheight\else\Gin@nat@height\fi}
\makeatother
% Scale images if necessary, so that they will not overflow the page
% margins by default, and it is still possible to overwrite the defaults
% using explicit options in \includegraphics[width, height, ...]{}
\setkeys{Gin}{width=\maxwidth,height=\maxheight,keepaspectratio}
% Set default figure placement to htbp
\makeatletter
\def\fps@figure{htbp}
\makeatother
\setlength{\emergencystretch}{3em} % prevent overfull lines
\providecommand{\tightlist}{%
  \setlength{\itemsep}{0pt}\setlength{\parskip}{0pt}}
\setcounter{secnumdepth}{-\maxdimen} % remove section numbering

\title{Introduction to R}
\author{Qin Wang or}
\date{07/19/2019}

\begin{document}
\maketitle

{
\setcounter{tocdepth}{2}
\tableofcontents
}
\hypertarget{installing-and-using-r}{%
\section{Installing and Using R}\label{installing-and-using-r}}

R is a free-to-use software that is very popular in statistical
computing. You can download R from \url{https://www.r-project.org}. The
latest version is 3.6.1. Another software that makes using R easier is
Rstudio, which is available at \url{https://www.rstudio.com}. You can
find many useful on-line tutorials that help you set-up these two
software. This guild is created using R Markdown
(\url{https://rmarkdown.rstudio.com/}), which is a feature provided by
RStudio.

\hypertarget{basic-mathematical-operations}{%
\section{Basic Mathematical
Operations}\label{basic-mathematical-operations}}

We will start with some basic operations. Try type-in the following
commands into your R console and start to explore yourself. Most of them
are self-explanatory. Lines with a \texttt{\#} in the front are
comments, which will not be executed. For displaying, lines with
\texttt{\#\#} in the front in the following chunk of code are outputs
from the previous line. This is a particular feature created by R
Markdown, which integrates text and code in a single document.
\vspace{12pt}

\begin{Shaded}
\begin{Highlighting}[]
    \CommentTok{# Basic mathematics}
    \DecValTok{1} \OperatorTok{+}\StringTok{ }\DecValTok{3}
\CommentTok{## [1] 4}
    \DecValTok{3}\OperatorTok{*}\DecValTok{5}
\CommentTok{## [1] 15}
    \DecValTok{3}\OperatorTok{^}\DecValTok{5}
\CommentTok{## [1] 243}
    \KeywordTok{exp}\NormalTok{(}\DecValTok{2}\NormalTok{)}
\CommentTok{## [1] 7.389056}
    \KeywordTok{log}\NormalTok{(}\DecValTok{3}\NormalTok{)}
\CommentTok{## [1] 1.098612}
    \KeywordTok{log2}\NormalTok{(}\DecValTok{3}\NormalTok{)}
\CommentTok{## [1] 1.584963}
    \KeywordTok{factorial}\NormalTok{(}\DecValTok{5}\NormalTok{)}
\CommentTok{## [1] 120}
\end{Highlighting}
\end{Shaded}

\vspace{12pt}

Most of our operations will be performed on data objects, such as
vectors and matrices. They can be created within R, loaded from existing
package or read-in from a file (to be discussed later on). For now,
let's create them in R. Note that for data objects, common mathematical
operations can be performed such as matrix multiplication, using the
\texttt{\%*\%} sign.

\begin{Shaded}
\begin{Highlighting}[]
    \CommentTok{# vector}
    \KeywordTok{c}\NormalTok{(}\DecValTok{1}\NormalTok{,}\DecValTok{2}\NormalTok{,}\DecValTok{3}\NormalTok{,}\DecValTok{4}\NormalTok{)}
\CommentTok{## [1] 1 2 3 4}
    \CommentTok{# matrix }
    \KeywordTok{matrix}\NormalTok{(}\KeywordTok{c}\NormalTok{(}\DecValTok{1}\NormalTok{,}\DecValTok{2}\NormalTok{,}\DecValTok{3}\NormalTok{,}\DecValTok{4}\NormalTok{), }\DecValTok{2}\NormalTok{, }\DecValTok{2}\NormalTok{)}
\CommentTok{##      [,1] [,2]}
\CommentTok{## [1,]    1    3}
\CommentTok{## [2,]    2    4}
    \CommentTok{# create matrix from vectors }
\NormalTok{    x =}\StringTok{ }\KeywordTok{c}\NormalTok{(}\DecValTok{1}\NormalTok{,}\DecValTok{1}\NormalTok{,}\DecValTok{1}\NormalTok{,}\DecValTok{0}\NormalTok{,}\DecValTok{0}\NormalTok{,}\DecValTok{0}\NormalTok{); y =}\StringTok{ }\KeywordTok{c}\NormalTok{(}\DecValTok{1}\NormalTok{,}\DecValTok{0}\NormalTok{,}\DecValTok{1}\NormalTok{,}\DecValTok{0}\NormalTok{,}\DecValTok{1}\NormalTok{,}\DecValTok{0}\NormalTok{)}
    \KeywordTok{cbind}\NormalTok{(x,y)}
\CommentTok{##      x y}
\CommentTok{## [1,] 1 1}
\CommentTok{## [2,] 1 0}
\CommentTok{## [3,] 1 1}
\CommentTok{## [4,] 0 0}
\CommentTok{## [5,] 0 1}
\CommentTok{## [6,] 0 0}
    \CommentTok{# matrix multiplication}
    \KeywordTok{matrix}\NormalTok{(}\KeywordTok{c}\NormalTok{(}\DecValTok{1}\NormalTok{,}\DecValTok{2}\NormalTok{,}\DecValTok{3}\NormalTok{,}\DecValTok{4}\NormalTok{), }\DecValTok{2}\NormalTok{, }\DecValTok{2}\NormalTok{) }\OperatorTok\StringTok{ }\KeywordTok{t}\NormalTok{(}\KeywordTok{cbind}\NormalTok{(x,y))}
\CommentTok{##      [,1] [,2] [,3] [,4] [,5] [,6]}
\CommentTok{## [1,]    4    1    4    0    3    0}
\CommentTok{## [2,]    6    2    6    0    4    0}
    \CommentTok{# some simple operations }
\NormalTok{    x[}\DecValTok{3}\NormalTok{]}
\CommentTok{## [1] 1}
\NormalTok{    x[}\DecValTok{2}\OperatorTok{:}\DecValTok{5}\NormalTok{]}
\CommentTok{## [1] 1 1 0 0}
    \KeywordTok{cbind}\NormalTok{(x,y)[}\DecValTok{1}\OperatorTok{:}\DecValTok{2}\NormalTok{, ]}
\CommentTok{##      x y}
\CommentTok{## [1,] 1 1}
\CommentTok{## [2,] 1 0}
\NormalTok{    x }\OperatorTok{+}\StringTok{ }\DecValTok{1}
\CommentTok{## [1] 2 2 2 1 1 1}
    \KeywordTok{length}\NormalTok{(x)}
\CommentTok{## [1] 6}
    \KeywordTok{dim}\NormalTok{(}\KeywordTok{cbind}\NormalTok{(x,y))}
\CommentTok{## [1] 6 2}
\end{Highlighting}
\end{Shaded}

\hypertarget{random-number-generations-from-a-distribution}{%
\section{Random Number Generations from a
Distribution}\label{random-number-generations-from-a-distribution}}

Random number generation is important for statistical simulation. R
provides functions to generate random observations from a given
distribution. For example, the normal distribution and the binomial
distribution. Usually, random number generation function starts with
`\texttt{r}' and then the distribution name. For example,
\texttt{rnorm()}, \texttt{rt()}, \texttt{rbinom()}, etc.

\begin{Shaded}
\begin{Highlighting}[]
    \CommentTok{# generate 10 Bernoulli random variables as a vector}
    \KeywordTok{rbinom}\NormalTok{(}\DataTypeTok{n=}\DecValTok{10}\NormalTok{, }\DataTypeTok{size =} \DecValTok{1}\NormalTok{, }\DataTypeTok{prob =} \FloatTok{0.5}\NormalTok{)}
\CommentTok{##  [1] 1 0 1 0 1 1 0 0 0 1}
    \CommentTok{# 10 random normally distributed variables}
    \KeywordTok{rnorm}\NormalTok{(}\DataTypeTok{n=}\DecValTok{10}\NormalTok{, }\DataTypeTok{mean =} \DecValTok{0}\NormalTok{, }\DataTypeTok{sd =} \DecValTok{2}\NormalTok{)}
\CommentTok{##  [1]  1.6964734  1.7197866 -1.2734493  1.6068620 -2.5319824 -0.4462294  0.4586598 -0.2274168  2.0479469 -0.3802007}
\end{Highlighting}
\end{Shaded}

We can also calculate the probability density function (pdf) and
cumulative distribution function (cdf) of random variables. For example,
the normal pdf and cdf function can be evaluated on a sequence of
points, which can be visualized on a plot.

\begin{Shaded}
\begin{Highlighting}[]
    \CommentTok{# generate a sequence of numbers from -2 to 2}
\NormalTok{    x =}\StringTok{ }\KeywordTok{seq}\NormalTok{(}\OperatorTok{-}\DecValTok{2}\NormalTok{, }\DecValTok{2}\NormalTok{, }\DataTypeTok{by =} \FloatTok{0.2}\NormalTok{)}
    \CommentTok{# calculate normal pdf on these points}
\NormalTok{    fx =}\StringTok{ }\KeywordTok{dnorm}\NormalTok{(x, }\DataTypeTok{mean =} \DecValTok{0}\NormalTok{, }\DataTypeTok{sd =} \DecValTok{1}\NormalTok{)}
    
    \CommentTok{# plot them in a figure}
    \KeywordTok{plot}\NormalTok{(x, fx, }\DataTypeTok{pch =} \DecValTok{16}\NormalTok{)}
\end{Highlighting}
\end{Shaded}

\begin{center}\includegraphics[width=0.5\linewidth]{00-r-intro_files/figure-latex/unnamed-chunk-4-1} \end{center}

\vspace{12pt}

If we need to replicate the results, we can set a random seed

\begin{Shaded}
\begin{Highlighting}[]
    \CommentTok{# after setting the seed, the random generation will follow a particular sequence}
    \KeywordTok{set.seed}\NormalTok{(}\DecValTok{697}\NormalTok{)}
    \KeywordTok{rnorm}\NormalTok{(}\DataTypeTok{n=}\DecValTok{4}\NormalTok{, }\DataTypeTok{mean =} \DecValTok{1}\NormalTok{, }\DataTypeTok{sd =} \DecValTok{2}\NormalTok{)}
\CommentTok{## [1] 1.4531482 3.5246932 3.2333684 0.1382969}
    
    \CommentTok{# if we don't reset the seed, a new set of random numbers will be generated }
    \KeywordTok{rnorm}\NormalTok{(}\DataTypeTok{n=}\DecValTok{4}\NormalTok{, }\DataTypeTok{mean =} \DecValTok{1}\NormalTok{, }\DataTypeTok{sd =} \DecValTok{2}\NormalTok{)}
\CommentTok{## [1] 2.702917 2.591680 3.371286 1.359250}
    
    \CommentTok{# if we rest the seed, we will replicate exactly the same results as the first vector}
    \KeywordTok{set.seed}\NormalTok{(}\DecValTok{697}\NormalTok{)}
    \KeywordTok{rnorm}\NormalTok{(}\DataTypeTok{n=}\DecValTok{4}\NormalTok{, }\DataTypeTok{mean =} \DecValTok{1}\NormalTok{, }\DataTypeTok{sd =} \DecValTok{2}\NormalTok{)}
\CommentTok{## [1] 1.4531482 3.5246932 3.2333684 0.1382969}
\end{Highlighting}
\end{Shaded}

\hypertarget{descriptive-statistics-of-data}{%
\section{Descriptive Statistics of
Data}\label{descriptive-statistics-of-data}}

\vspace{12pt}

Statistical functions that provide a summary of the data

\begin{Shaded}
\begin{Highlighting}[]
\NormalTok{    x =}\StringTok{ }\KeywordTok{rnorm}\NormalTok{(}\DataTypeTok{n=}\DecValTok{100}\NormalTok{, }\DataTypeTok{mean =} \DecValTok{1}\NormalTok{, }\DataTypeTok{sd =} \DecValTok{2}\NormalTok{)}
\NormalTok{    y =}\StringTok{ }\KeywordTok{rnorm}\NormalTok{(}\DataTypeTok{n=}\DecValTok{100}\NormalTok{, }\DataTypeTok{mean =} \DecValTok{2}\NormalTok{, }\DataTypeTok{sd =} \DecValTok{1}\NormalTok{)}
    \KeywordTok{sum}\NormalTok{(x)}
\CommentTok{## [1] 70.46658}
    \KeywordTok{mean}\NormalTok{(x)}
\CommentTok{## [1] 0.7046658}
    \KeywordTok{var}\NormalTok{(x)}
\CommentTok{## [1] 4.122945}
    \KeywordTok{median}\NormalTok{(x)}
\CommentTok{## [1] 0.8064754}
    \KeywordTok{quantile}\NormalTok{(x, }\KeywordTok{c}\NormalTok{(}\FloatTok{0.25}\NormalTok{, }\FloatTok{0.5}\NormalTok{, }\FloatTok{0.75}\NormalTok{))}
\CommentTok{##        25%        50%        75% }
\CommentTok{## -0.5204091  0.8064754  2.1611846}
    \KeywordTok{cor}\NormalTok{(x, y)}
\CommentTok{## [1] -0.004062204}
\end{Highlighting}
\end{Shaded}

\vspace{12pt}

For discrete data, we usually use the table function

\begin{Shaded}
\begin{Highlighting}[]
    \KeywordTok{set.seed}\NormalTok{(}\DecValTok{1}\NormalTok{); n =}\StringTok{ }\DecValTok{1000}
\NormalTok{    x =}\StringTok{ }\KeywordTok{rbinom}\NormalTok{(n, }\DataTypeTok{size =} \DecValTok{1}\NormalTok{, }\DataTypeTok{prob =} \FloatTok{0.75}\NormalTok{)}
\NormalTok{    y =}\StringTok{ }\KeywordTok{rbinom}\NormalTok{(n, }\DataTypeTok{size =} \DecValTok{3}\NormalTok{, }\DataTypeTok{prob =} \KeywordTok{c}\NormalTok{(}\FloatTok{0.4}\NormalTok{, }\FloatTok{0.3}\NormalTok{, }\FloatTok{0.2}\NormalTok{, }\FloatTok{0.1}\NormalTok{))}
    \KeywordTok{table}\NormalTok{(x)}
\CommentTok{## x}
\CommentTok{##   0   1 }
\CommentTok{## 248 752}
    
    \CommentTok{# this will be a cross table}
    \KeywordTok{table}\NormalTok{(x, y)}
\CommentTok{##    y}
\CommentTok{## x     0   1   2   3}
\CommentTok{##   0 128  79  34   7}
\CommentTok{##   1 342 267 125  18}
\end{Highlighting}
\end{Shaded}

For a mixture of discrete and continuous data (multiple variables), we
often use a data.frame

\begin{Shaded}
\begin{Highlighting}[]
    \CommentTok{# data.frame is a special data structure that can store both factor and numeric variables}
\NormalTok{    z =}\StringTok{ }\KeywordTok{runif}\NormalTok{(n, }\DataTypeTok{min =} \DecValTok{18}\NormalTok{, }\DataTypeTok{max =} \DecValTok{65}\NormalTok{)}
\NormalTok{    data =}\StringTok{ }\KeywordTok{data.frame}\NormalTok{(}\StringTok{"Gender"}\NormalTok{ =}\StringTok{ }\KeywordTok{as.factor}\NormalTok{(x), }\StringTok{"Group"}\NormalTok{ =}\StringTok{ }\KeywordTok{as.factor}\NormalTok{(y), }\StringTok{"Age"}\NormalTok{ =}\StringTok{ }\NormalTok{z)}
    \KeywordTok{levels}\NormalTok{(data}\OperatorTok{$}\NormalTok{Gender) =}\StringTok{ }\KeywordTok{c}\NormalTok{(}\StringTok{"male"}\NormalTok{, }\StringTok{"female"}\NormalTok{)}
    \KeywordTok{levels}\NormalTok{(data}\OperatorTok{$}\NormalTok{Group) =}\StringTok{ }\KeywordTok{c}\NormalTok{(}\StringTok{"patient"}\NormalTok{, }\StringTok{"physician"}\NormalTok{, }\StringTok{"engineer"}\NormalTok{, }\StringTok{"statistician"}\NormalTok{)}
    
    \CommentTok{# a peek at the top 3 entries of the data}
    \KeywordTok{head}\NormalTok{(data, }\DecValTok{3}\NormalTok{)}
\CommentTok{##   Gender     Group      Age}
\CommentTok{## 1 female physician 58.97484}
\CommentTok{## 2 female physician 63.45826}
\CommentTok{## 3 female   patient 58.74506}
    
    \CommentTok{# a brief summary}
    \KeywordTok{summary}\NormalTok{(data)}
\CommentTok{##     Gender             Group          Age       }
\CommentTok{##  male  :248   patient     :470   Min.   :18.03  }
\CommentTok{##  female:752   physician   :346   1st Qu.:29.07  }
\CommentTok{##               engineer    :159   Median :40.51  }
\CommentTok{##               statistician: 25   Mean   :41.02  }
\CommentTok{##                                  3rd Qu.:53.43  }
\CommentTok{##                                  Max.   :64.99}
\end{Highlighting}
\end{Shaded}

As a real example, we take the classical \texttt{iris} data, which can
be loaded directly from R.

\begin{Shaded}
\begin{Highlighting}[]
    \CommentTok{# read-in the data}
    \KeywordTok{data}\NormalTok{(iris)}
    
    \CommentTok{# a peek at the top observations }
    \KeywordTok{head}\NormalTok{(iris)}
\CommentTok{##   Sepal.Length Sepal.Width Petal.Length Petal.Width Species}
\CommentTok{## 1          5.1         3.5          1.4         0.2  setosa}
\CommentTok{## 2          4.9         3.0          1.4         0.2  setosa}
\CommentTok{## 3          4.7         3.2          1.3         0.2  setosa}
\CommentTok{## 4          4.6         3.1          1.5         0.2  setosa}
\CommentTok{## 5          5.0         3.6          1.4         0.2  setosa}
\CommentTok{## 6          5.4         3.9          1.7         0.4  setosa}

    \CommentTok{# number of observations for different species}
    \KeywordTok{table}\NormalTok{(iris}\OperatorTok{$}\NormalTok{Species)}
\CommentTok{## }
\CommentTok{##     setosa versicolor  virginica }
\CommentTok{##         50         50         50}
    
    \CommentTok{# suppose we are interested in the mean of first 4 variables by species}
    \KeywordTok{aggregate}\NormalTok{(iris[, }\DecValTok{1}\OperatorTok{:}\DecValTok{4}\NormalTok{], }\KeywordTok{list}\NormalTok{(iris}\OperatorTok{$}\NormalTok{Species), mean)}
\CommentTok{##      Group.1 Sepal.Length Sepal.Width Petal.Length Petal.Width}
\CommentTok{## 1     setosa        5.006       3.428        1.462       0.246}
\CommentTok{## 2 versicolor        5.936       2.770        4.260       1.326}
\CommentTok{## 3  virginica        6.588       2.974        5.552       2.026}
\end{Highlighting}
\end{Shaded}

\hypertarget{producing-figures}{%
\section{Producing Figures}\label{producing-figures}}

Data visualization is an important part for analysis. Some of the
commonly used techniques include the histograms, bar plot, scatter plot.
You can also customize those figures with different colors and shapes to
facilitate the visualization. We use the iris data as an example.

\begin{Shaded}
\begin{Highlighting}[]
    \CommentTok{# first, we produce a histogram of the Sepal.Length variable }
    \KeywordTok{hist}\NormalTok{(iris}\OperatorTok{$}\NormalTok{Sepal.Length, }\DataTypeTok{xlab =} \StringTok{"Sepal Length"}\NormalTok{)}
\end{Highlighting}
\end{Shaded}

\begin{center}\includegraphics[width=0.5\linewidth]{00-r-intro_files/figure-latex/unnamed-chunk-10-1} \end{center}

\begin{Shaded}
\begin{Highlighting}[]

    \CommentTok{# bar plot can be used to compare different variables, or the same variable in different groups}
    \CommentTok{# the following boxplot compares Sepal.Length across different species}
    \CommentTok{# you can easily color them}
    \KeywordTok{boxplot}\NormalTok{(Sepal.Length }\OperatorTok{~}\StringTok{ }\NormalTok{Species, }\DataTypeTok{data=}\NormalTok{iris, }\DataTypeTok{col=}\KeywordTok{c}\NormalTok{(}\StringTok{"indianred"}\NormalTok{, }\StringTok{"deepskyblue"}\NormalTok{, }\StringTok{"darkorange"}\NormalTok{))}
\end{Highlighting}
\end{Shaded}

\begin{center}\includegraphics[width=0.5\linewidth]{00-r-intro_files/figure-latex/unnamed-chunk-10-2} \end{center}

\begin{Shaded}
\begin{Highlighting}[]
    
    \CommentTok{# scatter plot is usually used for two variables}
    \CommentTok{# but we can color them using a third, categorical varaible}
    \CommentTok{# remeber to add legend for readability}
    \KeywordTok{plot}\NormalTok{(iris}\OperatorTok{$}\NormalTok{Sepal.Length, iris}\OperatorTok{$}\NormalTok{Sepal.Width, }\DataTypeTok{pch =} \DecValTok{19}\NormalTok{,}
         \DataTypeTok{col =} \KeywordTok{c}\NormalTok{(}\StringTok{"indianred"}\NormalTok{,}\StringTok{"deepskyblue"}\NormalTok{, }\StringTok{"darkorange"}\NormalTok{)[iris}\OperatorTok{$}\NormalTok{Species])}
    \KeywordTok{legend}\NormalTok{(}\StringTok{"topright"}\NormalTok{, }\KeywordTok{c}\NormalTok{(}\StringTok{"setosa"}\NormalTok{, }\StringTok{"versicolor"}\NormalTok{, }\StringTok{"virginica"}\NormalTok{), }
           \DataTypeTok{pch =} \DecValTok{19}\NormalTok{, }\DataTypeTok{col =} \KeywordTok{c}\NormalTok{(}\StringTok{"indianred"}\NormalTok{,}\StringTok{"deepskyblue"}\NormalTok{, }\StringTok{"darkorange"}\NormalTok{))}
\end{Highlighting}
\end{Shaded}

\begin{center}\includegraphics[width=0.5\linewidth]{00-r-intro_files/figure-latex/unnamed-chunk-10-3} \end{center}

\begin{Shaded}
\begin{Highlighting}[]
    
    \CommentTok{# to incorporate more than two variables, we can either use 3d plots}
    \CommentTok{# or do pairwise plots for each pair of variables}
    
    \KeywordTok{pairs}\NormalTok{(iris[,}\DecValTok{1}\OperatorTok{:}\DecValTok{4}\NormalTok{], }\DataTypeTok{pch =} \DecValTok{19}\NormalTok{,  }\DataTypeTok{cex =} \FloatTok{0.75}\NormalTok{, }\DataTypeTok{lower.panel=}\OtherTok{NULL}\NormalTok{,}
          \DataTypeTok{col =} \KeywordTok{c}\NormalTok{(}\StringTok{"indianred"}\NormalTok{,}\StringTok{"deepskyblue"}\NormalTok{, }\StringTok{"darkorange"}\NormalTok{)[iris}\OperatorTok{$}\NormalTok{Species])}
\end{Highlighting}
\end{Shaded}

\begin{center}\includegraphics[width=0.5\linewidth]{00-r-intro_files/figure-latex/unnamed-chunk-10-4} \end{center}

\hypertarget{read-in-and-save-data}{%
\section{Read-in and Save Data}\label{read-in-and-save-data}}

R can read-in data from many different sources such as \texttt{.txt},
\texttt{.csv}, etc. For example, \texttt{read.csv()} can be used to
import \texttt{.csv} files. The first argument should be specified as
the path to the data file, or just the name of the file if the current
working directory is the same as the data file. R objects, especially
matrices, can be saved into these standard files. Use functions such as
\texttt{write.table()} and \texttt{write.csv} to perform this. We can
also save any object into \texttt{.RData} file, which can be loaded
later on. To do this try functions \texttt{save.image()} and
\texttt{save()}.

\hypertarget{r-packages}{%
\section{R Packages}\label{r-packages}}

One of the most important features of R is its massive collection of
packages. A package is like an add-on that can be downloaded and
installed and perform additional function and analysis.

\begin{Shaded}
\begin{Highlighting}[]

    \CommentTok{# The MASS package can be used to generate multivariate normal distribution }
    \CommentTok{# This package is already included with the base R}
    \KeywordTok{library}\NormalTok{(MASS)}
\NormalTok{    P =}\StringTok{ }\DecValTok{4}\NormalTok{; N =}\StringTok{ }\DecValTok{200}
\NormalTok{    V <-}\StringTok{ }\FloatTok{0.5}\OperatorTok{^}\KeywordTok{abs}\NormalTok{(}\KeywordTok{outer}\NormalTok{(}\DecValTok{1}\OperatorTok{:}\NormalTok{P, }\DecValTok{1}\OperatorTok{:}\NormalTok{P, }\StringTok{"-"}\NormalTok{))}
\NormalTok{    X =}\StringTok{ }\KeywordTok{as.matrix}\NormalTok{(}\KeywordTok{mvrnorm}\NormalTok{(N, }\DataTypeTok{mu=}\KeywordTok{rep}\NormalTok{(}\DecValTok{0}\NormalTok{,P), }\DataTypeTok{Sigma=}\NormalTok{V))}
    \KeywordTok{head}\NormalTok{(X, }\DecValTok{3}\NormalTok{)}
\CommentTok{##            [,1]       [,2]       [,3]       [,4]}
\CommentTok{## [1,]  0.5170973 -0.5084187 -0.4949838 -2.0529670}
\CommentTok{## [2,]  0.4995059  0.5314248  0.7749723  0.8980893}
\CommentTok{## [3,] -1.9868469 -0.9742923 -1.1977524  1.8032596}
\end{Highlighting}
\end{Shaded}

Most packages need to be downloaded and installed. To do this, we use
the \texttt{install.packages()} function. After the package is loaded to
your console using \texttt{library()} we can start to utilize the
functions within that package.

\begin{Shaded}
\begin{Highlighting}[]
    \CommentTok{# we demonstrate using a package that can produce 3d images}

    \CommentTok{# to install the package}
    \KeywordTok{install.packages}\NormalTok{(}\StringTok{"scatterplot3d"}\NormalTok{)}
\end{Highlighting}
\end{Shaded}

Please note that you may not want to install a package every time the R
markdown file is compiled.

\begin{Shaded}
\begin{Highlighting}[]
    \CommentTok{# load the package}
    \KeywordTok{library}\NormalTok{(scatterplot3d) }

    \CommentTok{# now produce a 3d plot}
    \KeywordTok{scatterplot3d}\NormalTok{(iris[,}\DecValTok{1}\OperatorTok{:}\DecValTok{3}\NormalTok{], }\DataTypeTok{pch =} \DecValTok{19}\NormalTok{, }
                  \DataTypeTok{color=}\KeywordTok{c}\NormalTok{(}\StringTok{"indianred"}\NormalTok{,}\StringTok{"deepskyblue"}\NormalTok{, }\StringTok{"darkorange"}\NormalTok{)[iris}\OperatorTok{$}\NormalTok{Species])}
    \KeywordTok{legend}\NormalTok{(}\StringTok{"bottom"}\NormalTok{, }\DataTypeTok{legend =} \KeywordTok{levels}\NormalTok{(iris}\OperatorTok{$}\NormalTok{Species), }\DataTypeTok{pch =} \DecValTok{19}\NormalTok{,}
           \DataTypeTok{col =} \KeywordTok{c}\NormalTok{(}\StringTok{"indianred"}\NormalTok{,}\StringTok{"deepskyblue"}\NormalTok{, }\StringTok{"darkorange"}\NormalTok{),}
      \DataTypeTok{inset =} \FloatTok{-0.3}\NormalTok{, }\DataTypeTok{xpd =} \OtherTok{TRUE}\NormalTok{, }\DataTypeTok{horiz =} \OtherTok{TRUE}\NormalTok{)}
\end{Highlighting}
\end{Shaded}

\begin{center}\includegraphics[width=0.5\linewidth]{00-r-intro_files/figure-latex/unnamed-chunk-13-1} \end{center}

\hypertarget{basic-functions-for-statistics}{%
\section{Basic Functions for
Statistics}\label{basic-functions-for-statistics}}

For discrete variables, we can construct a frequency table to summerize
the data.

\begin{Shaded}
\begin{Highlighting}[]
    \CommentTok{# Summarizes Gender vs. Group}
    \KeywordTok{table}\NormalTok{(data[, }\DecValTok{1}\OperatorTok{:}\DecValTok{2}\NormalTok{])}
\CommentTok{##         Group}
\CommentTok{## Gender   patient physician engineer statistician}
\CommentTok{##   male       128        79       34            7}
\CommentTok{##   female     342       267      125           18}
\end{Highlighting}
\end{Shaded}

For continuous variables, we can calculate Pearson's correlation.

\begin{Shaded}
\begin{Highlighting}[]
    \KeywordTok{set.seed}\NormalTok{(}\DecValTok{1}\NormalTok{); n =}\StringTok{ }\DecValTok{30}
\NormalTok{    x =}\StringTok{ }\KeywordTok{rnorm}\NormalTok{(n)}
\NormalTok{    y =}\StringTok{ }\NormalTok{x }\OperatorTok{+}\StringTok{ }\KeywordTok{rnorm}\NormalTok{(n, }\DataTypeTok{sd =} \DecValTok{2}\NormalTok{)}
\NormalTok{    z =}\StringTok{ }\NormalTok{x }\OperatorTok{+}\StringTok{ }\KeywordTok{rnorm}\NormalTok{(n, }\DataTypeTok{sd =} \DecValTok{2}\NormalTok{)}
    
    \CommentTok{# Calculate Pearson's correlation and tests}
    \KeywordTok{cor}\NormalTok{(y, z)}
\CommentTok{## [1] 0.5810874}
    \KeywordTok{cor.test}\NormalTok{(y, z)}
\CommentTok{## }
\CommentTok{##  Pearson's product-moment correlation}
\CommentTok{## }
\CommentTok{## data:  y and z}
\CommentTok{## t = 3.7782, df = 28, p-value = 0.0007592}
\CommentTok{## alternative hypothesis: true correlation is not equal to 0}
\CommentTok{## 95 percent confidence interval:}
\CommentTok{##  0.2792861 0.7784002}
\CommentTok{## sample estimates:}
\CommentTok{##       cor }
\CommentTok{## 0.5810874}
\end{Highlighting}
\end{Shaded}

A t-test is often used to test the difference of a continuous variable
across two groups. For example, we test the Age differences between
genders.

\begin{Shaded}
\begin{Highlighting}[]
    \KeywordTok{t.test}\NormalTok{(data}\OperatorTok{$}\NormalTok{Age }\OperatorTok{~}\StringTok{ }\NormalTok{data}\OperatorTok{$}\NormalTok{Gender)}
\CommentTok{## }
\CommentTok{##  Welch Two Sample t-test}
\CommentTok{## }
\CommentTok{## data:  data$Age by data$Gender}
\CommentTok{## t = -0.92872, df = 407.2, p-value = 0.3536}
\CommentTok{## alternative hypothesis: true difference in means is not equal to 0}
\CommentTok{## 95 percent confidence interval:}
\CommentTok{##  -2.963039  1.061637}
\CommentTok{## sample estimates:}
\CommentTok{##   mean in group male mean in group female }
\CommentTok{##             40.30084             41.25154}
\end{Highlighting}
\end{Shaded}

A simple linear regression assumes the underlying model
\(Y = \beta X + \epsilon\). With observed data, we can estimate the
regression coefficients:

\begin{Shaded}
\begin{Highlighting}[]
    \CommentTok{# the lm() function is the most commonly used}
\NormalTok{    fit =}\StringTok{ }\KeywordTok{lm}\NormalTok{(y}\OperatorTok{~}\NormalTok{x)}
    \KeywordTok{summary}\NormalTok{(fit)}
\CommentTok{## }
\CommentTok{## Call:}
\CommentTok{## lm(formula = y ~ x)}
\CommentTok{## }
\CommentTok{## Residuals:}
\CommentTok{##     Min      1Q  Median      3Q     Max }
\CommentTok{## -3.0404 -1.0099 -0.4594  1.1506  3.7069 }
\CommentTok{## }
\CommentTok{## Coefficients:}
\CommentTok{##             Estimate Std. Error t value Pr(>|t|)   }
\CommentTok{## (Intercept)   0.2586     0.2964   0.873  0.39032   }
\CommentTok{## x             1.0838     0.3249   3.336  0.00241 **}
\CommentTok{## ---}
\CommentTok{## Signif. codes:  0 '***' 0.001 '**' 0.01 '*' 0.05 '.' 0.1 ' ' 1}
\CommentTok{## }
\CommentTok{## Residual standard error: 1.617 on 28 degrees of freedom}
\CommentTok{## Multiple R-squared:  0.2844, Adjusted R-squared:  0.2588 }
\CommentTok{## F-statistic: 11.13 on 1 and 28 DF,  p-value: 0.00241}
\end{Highlighting}
\end{Shaded}

\hypertarget{writing-your-own-functions}{%
\section{Writing Your Own Functions}\label{writing-your-own-functions}}

Writing your own R function can be an interesting experience. For
example, we can create a function that calculates the inner product of
two vectors:

\begin{Shaded}
\begin{Highlighting}[]
    \CommentTok{# the lm() function is the most commonly used}
\NormalTok{    myfun <-}\StringTok{ }\ControlFlowTok{function}\NormalTok{(a, b)}
\NormalTok{    \{}
      \KeywordTok{return}\NormalTok{(a }\OperatorTok\StringTok{ }\NormalTok{b)}
\NormalTok{    \}}

\NormalTok{    x1 =}\StringTok{ }\KeywordTok{rnorm}\NormalTok{(}\DecValTok{100}\NormalTok{)}
\NormalTok{    x2 =}\StringTok{ }\KeywordTok{rnorm}\NormalTok{(}\DecValTok{100}\NormalTok{)}
    \KeywordTok{myfun}\NormalTok{(x1, x2)}
\CommentTok{##           [,1]}
\CommentTok{## [1,] -5.537335}
\end{Highlighting}
\end{Shaded}

If we need the function to return multiple objects, we can create a
list. The following example returns both the inner product and the
correlation.

\begin{Shaded}
\begin{Highlighting}[]
    \CommentTok{# the lm() function is the most commonly used}
\NormalTok{    myfun <-}\StringTok{ }\ControlFlowTok{function}\NormalTok{(a, b)}
\NormalTok{    \{}
      \KeywordTok{return}\NormalTok{(}\KeywordTok{list}\NormalTok{(}\StringTok{"prod"}\NormalTok{ =}\StringTok{ }\NormalTok{a }\OperatorTok\StringTok{ }\NormalTok{b, }\StringTok{"cor"}\NormalTok{ =}\StringTok{ }\KeywordTok{cor}\NormalTok{(a, b)))}
\NormalTok{    \}}

    \KeywordTok{myfun}\NormalTok{(x1, x2)}
\CommentTok{## $prod}
\CommentTok{##           [,1]}
\CommentTok{## [1,] -5.537335}
\CommentTok{## }
\CommentTok{## $cor}
\CommentTok{## [1] -0.05558356}
\end{Highlighting}
\end{Shaded}

In fact, most functions in R are constructed this way be cause a lot of
information need to be stored when fitting a statistical model. You can
always peek into a fitted object of a model and take the information you
need.

\begin{Shaded}
\begin{Highlighting}[]
    \CommentTok{# the output is too long, hence omitted}
    \KeywordTok{str}\NormalTok{(fit)}
    \KeywordTok{names}\NormalTok{(fit)}
\end{Highlighting}
\end{Shaded}

\begin{Shaded}
\begin{Highlighting}[]
\NormalTok{    fit}\OperatorTok{$}\NormalTok{coefficients}
\CommentTok{## (Intercept)           x }
\CommentTok{##   0.2586414   1.0837726}
\end{Highlighting}
\end{Shaded}

\hypertarget{numerical-optimizations}{%
\section{Numerical Optimizations}\label{numerical-optimizations}}

Coefficients in a linear regression can be solved using the formula

\[\widehat \beta = (X'X)^{-1} X'Y.\]

We can also treat this as an optimization problem by minimizing the
\(\ell_2\) loss function:

\[\widehat \beta = \arg\min \frac{1}{2n}\lVert Y - X\beta \rVert_2^2.\]

This can be done through some numerical optimization methods
(\textcolor{red}{although this is entirely an overkill for the problem...}):

\begin{Shaded}
\begin{Highlighting}[]
\NormalTok{    f <-}\StringTok{ }\ControlFlowTok{function}\NormalTok{(b, X, Y)}
\NormalTok{    \{}
      \KeywordTok{length}\NormalTok{(Y)}\OperatorTok{^}\NormalTok{\{}\OperatorTok{-}\DecValTok{1}\NormalTok{\}}\OperatorTok{*}\FloatTok{0.5}\OperatorTok{*}\KeywordTok{sum}\NormalTok{((Y }\OperatorTok{-}\StringTok{ }\NormalTok{X }\OperatorTok\StringTok{ }\KeywordTok{as.matrix}\NormalTok{(b))}\OperatorTok{^}\DecValTok{2}\NormalTok{)}
\NormalTok{    \}}
    
    \CommentTok{# We use an optimization function: }
    \CommentTok{# the first argument is the inital value for beta}
    \CommentTok{# the second argument is the objective function}
    \CommentTok{# the rest of the arguments are additional information used}
    \CommentTok{# in calculating the objective function}

\NormalTok{    solution =}\StringTok{ }\KeywordTok{optim}\NormalTok{(}\KeywordTok{rep}\NormalTok{(}\DecValTok{0}\NormalTok{, }\DecValTok{2}\NormalTok{), f, }\DataTypeTok{X =} \KeywordTok{cbind}\NormalTok{(}\DecValTok{1}\NormalTok{, x), }\DataTypeTok{Y =}\NormalTok{ y)}
\NormalTok{    solution}\OperatorTok{$}\NormalTok{par}
\end{Highlighting}
\end{Shaded}

\begin{verbatim}
## [1] 0.2586419 1.0836772
\end{verbatim}

Many machine learning and statistical learning problems are eventually
an optimization problem, although they are much more difficult than
solving a linear regression.

\hypertarget{get-help}{%
\section{Get Help}\label{get-help}}

To get reference about a particular function, one can put a question
mark in front of a function name to see details:

\begin{Shaded}
\begin{Highlighting}[]
    \CommentTok{# This will open up the web browser for the on-line document }
\NormalTok{    ?mvrnorm}
\NormalTok{    ?save.image}
\end{Highlighting}
\end{Shaded}

\end{document}
